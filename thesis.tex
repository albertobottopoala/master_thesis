\documentclass[11pt]{article}
%Gummi|061|=)
\usepackage[english]{babel}
\usepackage[utf8]{inputenc}
\usepackage{hyperref}
\usepackage{graphicx}
\usepackage{amsmath}
\usepackage{frontespizio}

\DeclareMathOperator{\Tr}{Tr}
\DeclareMathOperator{\Exp}{exp}
\DeclareMathOperator{\Log}{log}
\DeclareMathOperator{\Det}{det}

\title{TITLE TITLE TITLE}
\author{Alberto Botto Poala\\
Advison Professor Friedrich Röpke}
\date{07.03.17}
\begin{document}
\begin{titlepage}
\begin{center}

% Upper part of the page. The '~' is needed because \\
% only works if a paragraph has started.
% to set spelling in Vim type :setlocal spell spelllang=en_us

\textsc{\LARGE Universität Heidelberg}\\[1.8cm]

\textsc{\Large Master Thesis}\\[0.5cm]

% Title

{ \huge \bfseries title title title title title \\[0.5cm] }



% Author and supervisor
\begin{minipage}{0.4\textwidth}
\begin{flushleft} \large
\emph{Candidate}\\
Alberto \textsc{Botto Poala}
\end{flushleft}
\end{minipage}
\begin{minipage}{0.4\textwidth}
\begin{flushright} \large
\emph{Advison Professor} \\
Prof. Friedrich \textsc{Röpke}
\end{flushright}
\end{minipage}

\vfill

% Bottom of the page
{07.03.17}

\end{center}
\end{titlepage}

\tableofcontents
\maketitle

%%%%%%%%%%%%%%%%%%%%%%%%%%%%%%%%%%%%%%%%%%%%%%%%%%%%%%%%%%%%%%%%%%%%%%%%%%%%%%%%%%%%%%%%%
% SECTION INTRODUCTION
%%%%%%%%%%%%%%%%%%%%%%%%%%%%%%%%%%%%%%%%%%%%%%%%%%%%%%%%%%%%%%%%%%%%%%%%%%%%%%%%%%%%%%%%%

\section{Introduction}
\subsection{What is convection?}
\subsubsection{Convection in Astrophysics}
convection is very important in astrophysics
\subsubsection{Convection in Geophysics}
convection is very important in geophysics too!!! 
\subsubsection{Convection Somewhere Else}
And somewhere else for sure!!!

%%%%%%%%%%%%%%%%%%%%%%%%%%%%%%%%%%%%%%%%%%%%%%%%%%%%%%%%%%%%%%%%%%%%%%%%%%%%%%%%%%%%%%%%%
% SECTION UNDERLYING PHYSICS
%%%%%%%%%%%%%%%%%%%%%%%%%%%%%%%%%%%%%%%%%%%%%%%%%%%%%%%%%%%%%%%%%%%%%%%%%%%%%%%%%%%%%%%%%

\section{Underlying Physics}
\subsection{Hydrodynamics}
\subsubsection{Derivation of the Equations of Ideal Hydrodynamics}
As all the most beautiful and successful laws of Physics, Hydrodynamics is derived from some conserved quantities. \\
Let's consider for instance a monoatomic gas in a box. Particles might be equally distributed in it (like in a bottle) or might not (maybe because the box is so big that we get a stratification because of gravity like the one in the atmosphere). Same story holds for the velocity distribution, it might be Maxwellian or it might not be.
In any case we can define a \textbf{distribution function in phase space} $f( \vec{x^{\mu}}, \vec{p^{\mu}})$ such that

$$N=\int d^4x d^4p f( x^{\mu}, p^{\mu}) \delta_D[(p^0)^2-\vec{p}^2+m^2c^2] \Theta(p^0)$$ 
 where $N$ is the total number of particle, $\delta_D$ is the Dirac $\delta$ function that selects in the integrations only the hypersurfaces physically allowed by the energy-momentum relation allowed by General Relativity, and finally $\Theta$ makes sure that we are taking into account only positive momenta. 
 namely the \textbf{current density} $J$ and the \textbf{energy-momentum tensor} $T$. These are nothing but the first and second momentum of the particle distribution function in the phase space. Carrying out the calculation the components read 


\[
  J= \frac{n(t,\vec{x})}{c}
  \begin{pmatrix}
    c \\
    \left< \dot{x} \right> \\
    \left< \dot{y} \right> \\
    \left< \dot{z} \right> \\
  \end{pmatrix}\quad
  T= \rho(t,\vec{x})
  \begin{pmatrix}
    c^2 \left< \gamma \right> & c \left< \gamma \dot{x} \right> &  c \left< \gamma \dot{y} \right> & c \left< \gamma \dot{z} \right> \\
     c \left< \gamma \dot{x} \right> & \left< \gamma \dot{x}^2 \right> &  \left< \gamma \dot{x}y \right> &  \left< \gamma \dot{x} \dot{z} \right>  \\
     c \left< \gamma \dot{y} \right> &  \left< \gamma \dot{y} \dot{x} \right>&  \left< \gamma \dot{y}^2 \right> &  \left< \gamma \dot{y} \dot{z} \right>  \\
     c \left< \gamma \dot{z} \right> &  \left< \gamma \dot{z}\dot{x} \right>& \left< \gamma \dot{z}\dot{y} \right> &  \left< \gamma \dot{z}^2 \right> 
  \end{pmatrix}
\]

where $n(t, \vec{x})$ is the number density, $c$ the speed of light,  $\left< a(t,\vec{x}) \right>$ means the average of the quantity $a$ over time and space in a neiborhood of $(t, \vec{x})$, $\gamma$ is the well known general relativistic parameter.\\
By integrating the first and second momentum of Boltzmann Equation, one could show that $J$ and $T$ are divergenceless, meaning in the non-relativistic case
\begin{align}
\frac{\partial J^{\mu}}{\partial^{\mu}}=0 && \frac{\partial T^{\mu \nu}}{\partial^{\mu}}=0 \  \  \forall \nu =0,..,3 
\end{align}
This means that we have to strike $J$ and $T$ with the operator $(c^{-1}\partial_t,\partial_x,\partial_y,\partial_z)$ from upside down. Doing so with the density current we obtain the \textbf{continuity equation}
$$\partial_t \rho + \vec\nabla \cdot (\rho \vec{v})=0$$
We apply the same procedure to the energy-momentum tensor.\\
We might choose to do it in the first column ($\nu=0$), and that would lead us to the \textbf{energy conservation equation}
$$\partial_t \epsilon + \vec \nabla \cdot (\epsilon \vec{v}) + P\vec \nabla \cdot \vec{v}=0$$
where $\epsilon$ is the internal energy and $P$ the pressure. These two variables, that were not esplicitally included in $T$, arise naturally by splitting up the microscopic velocity of the fluid $\left <  \dot{x} \right >$ into a mean macroscopic velocity $\vec{v}$ and a random velocity in the neiborhood of the mean one $\vec{u}$
$$\left <  \dot{x} \right >= \vec{v}  +  \vec{u} $$
and recalling that 
$$\epsilon = \frac{\rho}{2} \left <  u^2 \right > = \frac{3}{2} n k_B T= \frac{P}{2}$$
When we strike on the other three columns of $T$, we obtain the \textbf{momentum conservation equations} in the three spatial dimensions.
$$\partial_t \vec{v} + (\vec{v} \cdot \vec \nabla) \vec{v} + \frac{\vec \nabla P}{\rho}=0$$
These are ofthen called \textbf{Euler equations}.

\subsubsection{Viscous Hydrodynamics}
\subsubsection{Hydrostatic Equilibrium in Stars}
\subsection{Thermodynamics}
\subsection{Fundamental Equations of Stellar Structure}
\subsection{Bulk-Richardson Number}
\subsection{Il r'Hamiltoniana}
ciao

%%%%%%%%%%%%%%%%%%%%%%%%%%%%%%%%%%%%%%%%%%%%%%%%%%%%%%%%%%%%%%%%%%%%%%%%%%%%%%%%%%%%%%%%%
% SECTION RECENT WORKS 
%%%%%%%%%%%%%%%%%%%%%%%%%%%%%%%%%%%%%%%%%%%%%%%%%%%%%%%%%%%%%%%%%%%%%%%%%%%%%%%%%%%%%%%%%

\section{Recent Works}
\subsection{Il r e l'Hamiltoniana}
ciao

%%%%%%%%%%%%%%%%%%%%%%%%%%%%%%%%%%%%%%%%%%%%%%%%%%%%%%%%%%%%%%%%%%%%%%%%%%%%%%%%%%%%%%%%%
% SECTION CODE DESCRIPTION
%%%%%%%%%%%%%%%%%%%%%%%%%%%%%%%%%%%%%%%%%%%%%%%%%%%%%%%%%%%%%%%%%%%%%%%%%%%%%%%%%%%%%%%%%

\section{Code Description}
\subsection{ciao}
ciao

%%%%%%%%%%%%%%%%%%%%%%%%%%%%%%%%%%%%%%%%%%%%%%%%%%%%%%%%%%%%%%%%%%%%%%%%%%%%%%%%%%%%%%%%%
% SECTION CODE DESCRIPTION
%%%%%%%%%%%%%%%%%%%%%%%%%%%%%%%%%%%%%%%%%%%%%%%%%%%%%%%%%%%%%%%%%%%%%%%%%%%%%%%%%%%%%%%%%

\section{Results}
\subsection{ciao}
ciao
\end{document}
