\documentclass[11pt]{article}
%Gummi|061|=)
\usepackage[english]{babel}
\usepackage[utf8]{inputenc}
\usepackage{hyperref}
\usepackage{graphicx}
\usepackage{amsmath}
\usepackage{frontespizio}

\DeclareMathOperator{\Tr}{Tr}
\DeclareMathOperator{\Exp}{exp}
\DeclareMathOperator{\Log}{log}
\DeclareMathOperator{\Det}{det}

\title{TITLE TITLE TITLE}
\author{Alberto Botto Poala\\
Advison Professor Friedrich Röpke}
\date{07.03.17}
\begin{document}
\begin{titlepage}
\begin{center}

% Upper part of the page. The '~' is needed because \\
% only works if a paragraph has started.
% to set spelling in Vim type :setlocal spell spelllang=en_us

\textsc{\LARGE Universität Heidelberg}\\[1.8cm]

\textsc{\Large Master Thesis}\\[0.5cm]

% Title

{ \huge \bfseries title title title title title \\[0.5cm] }



% Author and supervisor
\begin{minipage}{0.4\textwidth}
\begin{flushleft} \large
\emph{Candidate}\\
Alberto \textsc{Botto Poala}
\end{flushleft}
\end{minipage}
\begin{minipage}{0.4\textwidth}
\begin{flushright} \large
\emph{Advison Professor} \\
Prof. Friedrich \textsc{Röpke}
\end{flushright}
\end{minipage}

\vfill

% Bottom of the page
{07.03.17}

\end{center}
\end{titlepage}

\tableofcontents
\maketitle

%%%%%%%%%%%%%%%%%%%%%%%%%%%%%%%%%%%%%%%%%%%%%%%%%%%%%%%%%%%%%%%%%%%%%%%%%%%%%%%%%%%%%%%%%
% SECTION INTRODUCTION
%%%%%%%%%%%%%%%%%%%%%%%%%%%%%%%%%%%%%%%%%%%%%%%%%%%%%%%%%%%%%%%%%%%%%%%%%%%%%%%%%%%%%%%%%

\section{Introduction}
\subsection{What is convection?}
\subsubsection{Convection in Astrophysics}
convection is very important in astrophysics
\subsubsection{Convection in Geophysics}
convection is very important in geophysics too!!! 
\subsubsection{Convection Somewhere Else}
And somewhere else for sure!!!

%%%%%%%%%%%%%%%%%%%%%%%%%%%%%%%%%%%%%%%%%%%%%%%%%%%%%%%%%%%%%%%%%%%%%%%%%%%%%%%%%%%%%%%%%
% SECTION UNDERLYING PHYSICS
%%%%%%%%%%%%%%%%%%%%%%%%%%%%%%%%%%%%%%%%%%%%%%%%%%%%%%%%%%%%%%%%%%%%%%%%%%%%%%%%%%%%%%%%%

\section{Underlying Physics}
\subsection{Hydrodynamics}
\subsubsection{Derivation of the Equations of Ideal Hydrodynamics}
As all the most beautiful and successful laws of Physics, Hydrodynamics is derived from some conserved quantities. \\
Let's consider for instance a monoatomic gas in a box. A fundamental assumption in ideal hydrodynamic is that the mean free path of particles is infinitesimal. Particles might be equally distributed in it (like in a bottle) or might not (maybe because the box is so big that we get a stratification because of gravity like the one in the atmosphere). Same story holds for the velocity distribution, it might be Maxwellian or it might not be. 
In any case we can define a \textbf{distribution function in phase space} $f( \vec{x^{\mu}}, \vec{p^{\mu}})$ such that

$$N=\int \ d^4x \ d^4p \ f( x^{\mu}, p^{\mu}) \  \delta_D[(p^0)^2-\vec{p}^2+m^2c^2] \  \Theta(p^0)$$ 
 where $N$ is the total number of particle, $\delta_D$ is the Dirac $\delta$ function that selects in the integrations only the hypersurfaces physically allowed by the energy-momentum relation of General Relativity, and finally $\Theta$ makes sure that we are taking into account only positive momenta. 
Keeping in mind this, we can define the first and second momentum of our distribution function

\begin{align}
J^{\alpha}= c \int \frac{dp^3}{E} \ f( x^{\mu}, p^{\mu}) p^{\alpha}  &&   T^{\alpha \beta}= c \int \frac{dp^3}{E} \ f( x^{\mu}, p^{\mu}) p^{\alpha}p^{\beta}  
\end{align}
 
namely the \textbf{current density} $J$ and the \textbf{energy-momentum tensor} $T$. Carrying out the calculation the components read 


\[
  J= \frac{n(t,\vec{x})}{c}
  \begin{pmatrix}
    c \\
    \left< \dot{x} \right> \\
    \left< \dot{y} \right> \\
    \left< \dot{z} \right> \\
  \end{pmatrix}\quad
  T= \rho(t,\vec{x})
  \begin{pmatrix}
    c^2 \left< \gamma \right> & c \left< \gamma \dot{x} \right> &  c \left< \gamma \dot{y} \right> & c \left< \gamma \dot{z} \right> \\
     c \left< \gamma \dot{x} \right> & \left< \gamma \dot{x}^2 \right> &  \left< \gamma \dot{x}y \right> &  \left< \gamma \dot{x} \dot{z} \right>  \\
     c \left< \gamma \dot{y} \right> &  \left< \gamma \dot{y} \dot{x} \right>&  \left< \gamma \dot{y}^2 \right> &  \left< \gamma \dot{y} \dot{z} \right>  \\
     c \left< \gamma \dot{z} \right> &  \left< \gamma \dot{z}\dot{x} \right>& \left< \gamma \dot{z}\dot{y} \right> &  \left< \gamma \dot{z}^2 \right> 
  \end{pmatrix}
\]

where $n(t, \vec{x})$ is the number density, $c$ the speed of light,  $\left< a(t,\vec{x}) \right>$ means the average of the quantity $a$ over time and space in a neighborhood of $(t, \vec{x})$, $\gamma$ is the well known general relativistic parameter.\\
By integrating the first and second momentum of Boltzmann Equation, one could show that $J$ and $T$ are divergenceless, meaning in the non-relativistic case
\begin{align}
\frac{\partial J^{\mu}}{\partial^{\mu}}=0 && \frac{\partial T^{\mu \nu}}{\partial^{\mu}}=0 \  \  \forall \nu =0,..,3 
\end{align}
This means that we have to strike $J$ and $T$ with the operator $(c^{-1}\partial_t,\partial_x,\partial_y,\partial_z)$ from upside down. Doing so with the density current we obtain the \textbf{continuity equation}
\begin{equation} \label{cont}
\partial_t \rho + \vec\nabla \cdot (\rho \vec{v})=0
\end{equation}
We apply the same procedure to the energy-momentum tensor.\\
We might choose to do it in the first column ($\nu=0$), and that would lead us to the \textbf{energy conservation equation}
\begin{equation} \label{consen}
\partial_t \epsilon + \vec \nabla \cdot (\epsilon \vec{v}) + P\vec \nabla \cdot \vec{v}=0
\end{equation}
where $\epsilon$ is the internal energy and $P$ the pressure. These two variables, that were not esplecitally included in $T$, arise naturally by splitting up the microscopic velocity of the fluid $\left <  \dot{x} \right >$ into a mean macroscopic velocity $\vec{v}$ and a random velocity in the neighborhood of the mean one $\vec{u}$
$$\left <  \dot{x} \right >= \vec{v}  +  \vec{u} $$
and recalling that 
$$\epsilon = \frac{\rho}{2} \left <  u^2 \right > = \frac{3}{2} n k_B T= \frac{P}{2}$$
When we strike on the other three columns of $T$, we obtain the \textbf{momentum conservation equations} in the three spatial dimensions.
\begin{equation} \label{euler}
\partial_t \vec{v} + (\vec{v} \cdot \vec \nabla) \vec{v} + \frac{\vec \nabla P}{\rho}=0
\end{equation}
These are often called \textbf{Euler equations}. \\
Note that the operator on the left hand side is nothing but the total time derivative
$$
\partial_t + \vec{v} \cdot \vec \nabla = \frac{d}{dt}
$$
so what we are actually writing is Newton's equation per unit mass
$$
\rho \vec{a} = \vec \nabla P
$$
In case other macroscopical forces like gravity are present, we simply add them on the right hand side as we would do in classical mechanics. \\
So far we have obtained 5 equations in 6 unknowns, namely the internal energy $\epsilon$, the momentum $\vec{p}$, the pressure $P$ and the density $\rho$. If we want to have at least a chance of integrating we're missing one equation, the \textbf{equation of state}, that relates the thermodinamical variables.

\subsubsection{Viscous Hydrodynamics}

As stated at the beginning of the previous subsection, one fundamental assumption of ideal Hydrodynamics is that particles have an infinitesimal mean free path. What actually happens in nature might be very different. Particle have a finite mean free path, and hence they can transport local property of the fluid by diffusion only: transport of energy generates heat conduction, transport of momentum friction. The consecuence is that we end up with one additional term in the energy-momentum tensor representing the diffusive processes
$$
T^{ij} \to T^{ij} + T^{ij}_d 
$$

while the density current remains unchanged because mass is always conserved. As in the previous subsection, the new $T$ needs to be striked with $\nabla$ and set equal to zero in order to obtain our modified equations in the viscous case. We will not carry out all the calculations, rather simply quote the result for the new QUOTE and QUOTE. \\
The new energy cinservation equation reads
$$
\partial_t \epsilon + \vec \nabla \cdot (\epsilon \vec{v}) + P \vec \nabla \cdot \vec{v} = \vec \nabla \cdot (k \vec \nabla T) + v_{ij} T_d^{ij} 
$$
where $v_{ij}$ is the symmetrised velocity gradient tensor
$$
v_{ij} = \frac{1}{2}(\partial_i v_j + \partial_j v_i)
$$
This equation shows that the internal energy in a certain region of the fluid can change either if there is a temperature gradient or if the fluid is moving with a velovity field with a nonvanishing symmetrised gradient (non solid body rotation like) because of friction.
The new momentum conservation equations read
$$
\rho \left( \partial_t + \vec{v} \cdot \vec \nabla \right) v^j+\partial^jP = \eta \vec \nabla^2v^j + \left( \xi + \frac{\eta}{3} \right) \partial^j \vec \nabla \cdot \vec{v}
$$
These are often called \textbf{Navier-Stokes Equations}, and fully show the nonlinearity of Hydrodinamic.

\subsubsection{Hydrostatic Equilibrium in Stars}
We shall now consider the static configuration of QUOTE with gravitational potential. This reads
\begin{equation} \label{hystat}
	\vec \nabla P = - \rho \vec \nabla \Phi
\end{equation}
This shows that the pressure stratification is adjusted only by the gravitational potential. We can say more by taking the curl of this equation and we get
$$
\vec \nabla \rho \times  \vec \nabla \Phi =0
$$
which shows that the gradient of the gravitational potetial and of the density are parallel. Surfaces of constant density are surfaces of constant gravitational potential.\\
Taking the divergence of \ref{hystat} and substituting Poisson equation we can find the relation between pressure and density
$$
\vec \nabla \cdot \left ( \frac{\vec \nabla P}{\rho} \right ) = - 4 \pi G \rho 
$$
provided an equation of state.\\
In a spherical symmetry configuration exploiting the $\vec \nabla$ we obtain the \textbf{Lane-Emden Equation}. If we choose a polytropic equation of state, which means that the pressure is a function of the density only and not of the temperature, we obtain stable solutions for adiabatic indices up to $4/3$ only. This EoS is used for instance for modeling white dwarfs, where the pressure doesn't depend on the temperature. \\ 

\subsection{Thermodynamics}
\subsection{Fundamental Equations of Stellar Structure}
At this point we are able to write down a system of equation that we will call \textbf{fundamental equations of stellar structure}, which describe an oversimplified thus very useful model. This will hold in a static, stable and spherically symmetric case.\\

\textbf{Mass continuity} \\
Here $m(r)$ is the mass contained inside the sphere of radius $r$
\begin{equation}\label{masscons}
	\frac{dm}{dr}=4 \pi r^2 \rho
\end{equation}

\textbf{Hydrostatic equilibrium} \\
We have already encountered the hydrostatic equilibrium equation
\begin{equation}\label{hydroeq}
	\frac{dP}{dr}= - \frac{G m(r)}{r^2} \rho
\end{equation}

\textbf{Equation of State} \\
So far we have written down two equations in three variables, namely $m$, $\rho$, and $P$. If we want to have at lest a chanche to solve them we need another equation, namely the equation of state. Possible choices are:
\begin{itemize}
	\item $\rho=const$ this would be the case for liquids.
	\item $P\propto \rho^\gamma$ i. e. the density and pressure are not a function of the temperature. This is the case, as already stated in the previous section for white dwarfs and leads to the Lane-Edmen equation and its solutions.
	\item $P=\frac{\mu}{R}\frac{P}{T}$ the equation of state for perfect gases and it's the case for the sun. It has the inconvenient that it introduces another variable, namely the temperature $T$, and hence it doesn't solve our original problem. We need at least another equation to solve the system. 
\end{itemize}

\textbf{Energy conservation} \\
In case we have an energy source (such as nuclear reactions), this needs to be conserved
\begin{equation}\label{energycons}
	\frac{dL}{dr} = 4 \pi r^2 \epsilon
\end{equation}

\textbf{Energy transfer by radiation} \\
Energy can be radiated through a medium, depending on the opacity $k$.
\begin{equation}\label{energytransfer}
	L(r)=-\frac{c(4 \pi r^2)^2}{3k} \frac{d(aT^4)}{dr}
\end{equation}


\subsection{Convection}
Until now we have assumed a perfect spherical symmetry, meaning that all dynamical and thermodynamical quantities were a function of the radius only. Obviously this is not a realistic case. For a number of reasons stars have small perturbations, that may eventually grow and give rise to macroscopic phenomena. A classic example is convection.\\
We are relaxing the spherically symmetric case, but this doesn't mean that our fundamental equations of stellar structure are unuseful. As explained in the introdiction, if we could section stars we would see that convection appears in concentric regions, hence  we can still treat the problem as spherically symmetric defining dynamicam and thermodynamical quantities as averages on a proper region. \\
We shall now understand when, given thermodinamical variables, we obtain a dynamically stable or unstable layer.


\subsubsection{Dynamical instability}
Let's consider a perfect density stratification like the one in a star or in the atmosphere and let's break the symmetry of the system by adding a little perturbation in the termodynamical variables. For any given quantity $A(r, \theta)$ (from now on $A_e$, because it's a function of the mass element we are considering), we compute $A_s$ (which is an average at given $r$ on the surrounding material). We shall furthermore assume that the fluid moves adiabatically, thus the timescale for heating transfer is much smaller than the timescale for convection turnover. \\
We define a local property of the fluid 
$$
DA=A_e - A_s
$$
For instance we could imagine that a little region of a star is slighty hotter than the surroundings, hence we have in that region $DT > 0$. Note that because of the assumprion we have made $DP=0$ always, because when these is a pressure spike, the gas expands at the sound velocity $c_s$ which is way higher than the low hydro regime we observe in stellar interia.\\
If we have a $DT>0$, since in a perfect gas the equation of state reads $\rho \sim P/T$, we obtain $D \rho < 0$. With a lower density, that mass element will be lifted by buoyancy force. In a non adiabatic workframe it might happen that heat is exchanged so quickly that temerature differences vanish immediately, but we assumed adiabatic processes. The question we are trying to answer is if the mass element, after a little upward movement, will still be buoyant and give rise to macroscopic convection, or if it will bounce back. The answer lyes obviously in the temperature gradient, i. e. if once lifted of a little bit the new $DT$ will still be in favor to buoy it to the next layer, and so on. \\
Let's approache the problem from another point of view. Let's assume that we have a stable layer without perturbations, and we lift a mass element upward of $\Delta r$. The density difference now is
\begin{equation}\label{displacment}
D \rho = \left [  \left( \frac{d \rho}{d r} \right)_e - \left( \frac{d \rho}{d r} \right)_s   \right ] \Delta r
\end{equation}
where the first derivative tells us how much the density of the mass element changes when lifted, the second one tells us how the surrounding density changes along the radial direction. \\
We call the buoyancy force per unit volume 
$$
f_b=- g \ D \rho
$$
which points upward if $D \rho < 0$, which is the unstable configuration. If instead $D \rho > 0$ the mass element sinks back to its original position and no macroscopic motion appears. As a consequence $D \rho<0$ is our \textbf{condition for stability}.\\
The problem with this criterion is that very often the density gradient is not known, since it does not appear in the fundamental equations for stellar structure. In order to proceed let's turn our gradient in spacial coordinate into a gradient in thermodynamic coordinates. As previously stated, our transformations are adiabatic, hence no exchange of energy occurs. This is very close to reality for stellar interia. To begin let's write down the equation of state $\rho = \rho (P, T, \mu)$ in a differential form
\begin{equation}\label{EoSdiff}
	\frac{d \rho}{\rho} = \alpha \frac{d P}{P} - \delta \frac{d T}{T} + \phi \frac{d \mu}{\mu}
\end{equation}

\subsubsection{Bulk Richardson}

\subsection{Il r'Hamiltoniana}
ciao

%%%%%%%%%%%%%%%%%%%%%%%%%%%%%%%%%%%%%%%%%%%%%%%%%%%%%%%%%%%%%%%%%%%%%%%%%%%%%%%%%%%%%%%%%
% SECTION RECENT WORKS 
%%%%%%%%%%%%%%%%%%%%%%%%%%%%%%%%%%%%%%%%%%%%%%%%%%%%%%%%%%%%%%%%%%%%%%%%%%%%%%%%%%%%%%%%%

\section{Recent Works}
\subsection{Il r e l'Hamiltoniana}
ciao

%%%%%%%%%%%%%%%%%%%%%%%%%%%%%%%%%%%%%%%%%%%%%%%%%%%%%%%%%%%%%%%%%%%%%%%%%%%%%%%%%%%%%%%%%
% SECTION CODE DESCRIPTION
%%%%%%%%%%%%%%%%%%%%%%%%%%%%%%%%%%%%%%%%%%%%%%%%%%%%%%%%%%%%%%%%%%%%%%%%%%%%%%%%%%%%%%%%%

\section{Code Description}
\subsection{ciao}
ciao

%%%%%%%%%%%%%%%%%%%%%%%%%%%%%%%%%%%%%%%%%%%%%%%%%%%%%%%%%%%%%%%%%%%%%%%%%%%%%%%%%%%%%%%%%
% SECTION CODE DESCRIPTION
%%%%%%%%%%%%%%%%%%%%%%%%%%%%%%%%%%%%%%%%%%%%%%%%%%%%%%%%%%%%%%%%%%%%%%%%%%%%%%%%%%%%%%%%%

\section{Results}
\subsection{ciao}
ciao
\end{document}
