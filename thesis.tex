\documentclass[11pt]{article}
%Gummi|061|=)
\usepackage[english]{babel}
\usepackage[utf8]{inputenc}
\usepackage{hyperref}
\usepackage{graphicx}
\usepackage{amsmath}
\usepackage{frontespizio}

\DeclareMathOperator{\Tr}{Tr}
\DeclareMathOperator{\Exp}{exp}
\DeclareMathOperator{\Log}{log}
\DeclareMathOperator{\Det}{det}

\title{TITLE TITLE TITLE}
\author{Alberto Botto Poala\\
Advison Professor Friedrich Röpke}
\date{07.03.17}
\begin{document}
\begin{titlepage}
\begin{center}

% Upper part of the page. The '~' is needed because \\
% only works if a paragraph has started.
% to set spelling in Vim type :setlocal spell spelllang=en_us

\textsc{\LARGE Universität Heidelberg}\\[1.8cm]

\textsc{\Large Master Thesis}\\[0.5cm]

% Title

{ \huge \bfseries title title title title title \\[0.5cm] }



% Author and supervisor
\begin{minipage}{0.4\textwidth}
\begin{flushleft} \large
\emph{Candidate}\\
Alberto \textsc{Botto Poala}
\end{flushleft}
\end{minipage}
\begin{minipage}{0.4\textwidth}
\begin{flushright} \large
\emph{Advison Professor} \\
Prof. Friedrich \textsc{Röpke}
\end{flushright}
\end{minipage}

\vfill

% Bottom of the page
{07.03.17}

\end{center}
\end{titlepage}

\tableofcontents
\maketitle

%%%%%%%%%%%%%%%%%%%%%%%%%%%%%%%%%%%%%%%%%%%%%%%%%%%%%%%%%%%%%%%%%%%%%%%%%%%%%%%%%%%%%%%%%
% SECTION INTRODUCTION
%%%%%%%%%%%%%%%%%%%%%%%%%%%%%%%%%%%%%%%%%%%%%%%%%%%%%%%%%%%%%%%%%%%%%%%%%%%%%%%%%%%%%%%%%

\section{Introduction}
\subsection{What is convection?}
\subsubsection{Convection in Astrophysics}
convection is very important in astrophysics
\subsubsection{Convection in Geophysics}
convection is very important in geophysics too!!! 
\subsubsection{Convection Somewhere Else}
And somewhere else for sure!!!

%%%%%%%%%%%%%%%%%%%%%%%%%%%%%%%%%%%%%%%%%%%%%%%%%%%%%%%%%%%%%%%%%%%%%%%%%%%%%%%%%%%%%%%%%
% SECTION UNDERLYING PHYSICS
%%%%%%%%%%%%%%%%%%%%%%%%%%%%%%%%%%%%%%%%%%%%%%%%%%%%%%%%%%%%%%%%%%%%%%%%%%%%%%%%%%%%%%%%%

\section{Underlying Physics}
\subsection{Hydrodynamics}
\subsubsection{Derivation of the Equations of Ideal Hydrodynamics}
As all the most beautiful and successful laws of Physics, Hydrodynamics is derived from some conserved quantities. \\
Let's consider for instance a monoatomic gas in a box. A fundamental assumption in ideal hydrodynamic is that the mean free path of particles is infinitesimal. Particles might be equally distributed in it (like in a bottle) or might not (maybe because the box is so big that we get a stratification because of gravity like the one in the atmosphere). Same story holds for the velocity distribution, it might be Maxwellian or it might not be. 
In any case we can define a \textbf{distribution function in phase space} $f( \vec{x^{\mu}}, \vec{p^{\mu}})$ such that

$$N=\int \ d^4x \ d^4p \ f( x^{\mu}, p^{\mu}) \  \delta_D[(p^0)^2-\vec{p}^2+m^2c^2] \  \Theta(p^0)$$ 
 where $N$ is the total number of particle, $\delta_D$ is the Dirac $\delta$ function that selects in the integrations only the hypersurfaces physically allowed by the energy-momentum relation of General Relativity, and finally $\Theta$ makes sure that we are taking into account only positive momenta. 
Keeping in mind this, we can define the first and second momentum of our distribution function

\begin{align}
J^{\alpha}= c \int \frac{dp^3}{E} \ f( x^{\mu}, p^{\mu}) p^{\alpha}  &&   T^{\alpha \beta}= c \int \frac{dp^3}{E} \ f( x^{\mu}, p^{\mu}) p^{\alpha}p^{\beta}  
\end{align}
 
namely the \textbf{current density} $J$ and the \textbf{energy-momentum tensor} $T$. Carrying out the calculation the components read 


\[
  J= \frac{n(t,\vec{x})}{c}
  \begin{pmatrix}
    c \\
    \left< \dot{x} \right> \\
    \left< \dot{y} \right> \\
    \left< \dot{z} \right> \\
  \end{pmatrix}\quad
  T= \rho(t,\vec{x})
  \begin{pmatrix}
    c^2 \left< \gamma \right> & c \left< \gamma \dot{x} \right> &  c \left< \gamma \dot{y} \right> & c \left< \gamma \dot{z} \right> \\
     c \left< \gamma \dot{x} \right> & \left< \gamma \dot{x}^2 \right> &  \left< \gamma \dot{x}y \right> &  \left< \gamma \dot{x} \dot{z} \right>  \\
     c \left< \gamma \dot{y} \right> &  \left< \gamma \dot{y} \dot{x} \right>&  \left< \gamma \dot{y}^2 \right> &  \left< \gamma \dot{y} \dot{z} \right>  \\
     c \left< \gamma \dot{z} \right> &  \left< \gamma \dot{z}\dot{x} \right>& \left< \gamma \dot{z}\dot{y} \right> &  \left< \gamma \dot{z}^2 \right> 
  \end{pmatrix}
\]

where $n(t, \vec{x})$ is the number density, $c$ the speed of light,  $\left< a(t,\vec{x}) \right>$ means the average of the quantity $a$ over time and space in a neighborhood of $(t, \vec{x})$, $\gamma$ is the well known general relativistic parameter.\\
By integrating the first and second momentum of Boltzmann Equation, one could show that $J$ and $T$ are divergenceless, meaning in the non-relativistic case
\begin{align}
\frac{\partial J^{\mu}}{\partial^{\mu}}=0 && \frac{\partial T^{\mu \nu}}{\partial^{\mu}}=0 \  \  \forall \nu =0,..,3 
\end{align}
This means that we have to strike $J$ and $T$ with the operator $(c^{-1}\partial_t,\partial_x,\partial_y,\partial_z)$ from upside down. Doing so with the density current we obtain the \textbf{continuity equation}
\begin{equation}
\partial_t \rho + \vec\nabla \cdot (\rho \vec{v})=0
\end{equation}
We apply the same procedure to the energy-momentum tensor.\\
We might choose to do it in the first column ($\nu=0$), and that would lead us to the \textbf{energy conservation equation}
\begin{equation}
\partial_t \epsilon + \vec \nabla \cdot (\epsilon \vec{v}) + P\vec \nabla \cdot \vec{v}=0
\end{equation}
where $\epsilon$ is the internal energy and $P$ the pressure. These two variables, that were not esplecitally included in $T$, arise naturally by splitting up the microscopic velocity of the fluid $\left <  \dot{x} \right >$ into a mean macroscopic velocity $\vec{v}$ and a random velocity in the neighborhood of the mean one $\vec{u}$
$$\left <  \dot{x} \right >= \vec{v}  +  \vec{u} $$
and recalling that 
$$\epsilon = \frac{\rho}{2} \left <  u^2 \right > = \frac{3}{2} n k_B T= \frac{P}{2}$$
When we strike on the other three columns of $T$, we obtain the \textbf{momentum conservation equations} in the three spatial dimensions.
\begin{equation}
\partial_t \vec{v} + (\vec{v} \cdot \vec \nabla) \vec{v} + \frac{\vec \nabla P}{\rho}=0
\end{equation}
These are often called \textbf{Euler equations}. \\
So far we have obtained 5 equations in 6 unknowns, namely the internal energy $\epsilon$, the momentum $\vec{p}$, the pressure $P$ and the density $\rho$. If we want to have at least a chance of integrating we're missing one equation, the \textbf{equation of state}, that relates the thermodinamical variables.

\subsubsection{Viscous Hydrodynamics}

As stated at the beginning of the previous subsection, one fundamental assumption of ideal Hydrodynamics is that particles have an infinitesimal mean free path. What actually happens in nature might be very different. Particle have a finite mean free path, and hence they can transport local property of the fluid by diffusion only: transport of energy generates heat conduction, transport of momentum friction. The consecuence is that we end up with one additional term in the energy-momentum tensor representing the diffusive processes
$$
T^{ij} \to T^{ij} + T^{ij}_d 
$$

while the density current remains unchanged because mass is always conserved. As in the previous subsection, the new $T$ needs to be striked with $\nabla$ and set equal to zero in order to obtain our modified equations in the viscous case. We will not carry out all the calculations, rather simply quote the result for the new QUOTE and QUOTE. \\
The new energy cinservation equation reads
$$
\partial_t \epsilon + \vec \nabla \cdot (\epsilon \vec{v}) + P \vec \nabla \cdot \vec{v} = \vec \nabla \cdot (k \vec \nabla T) + v_{ij} T_d^{ij} 
$$
where $v_{ij}$ is the symmetrised velocity gradient tensor
$$
v_{ij} = \frac{1}{2}(\partial_i v_j + \partial_j v_i)
$$
This equation shows that the internal energy in a certain region of the fluid can change either if there is a temperature gradient or if the fluid is moving with a velovity field with a nonvanishing symmetrised gradient (non solid body rotation like) because of friction.
The new momentum conservation equations read
$$
\rho \left( \partial_t + \vec{v} \cdot \vec \nabla \right) v^j+\partial^jP = \eta \vec \nabla^2v^j + \left( \xi + \frac{\eta}{3} \right) \partial^j \vec \nabla \cdot \vec{v}
$$
These are often called \textbf{Navier-Stokes Equations}, and fully show the nonlinearity of Hydrodinamic.

\subsubsection{Hydrostatic Equilibrium in Stars}
\subsection{Thermodynamics}
\subsection{Fundamental Equations of Stellar Structure}
\subsection{Bulk-Richardson Number}
\subsection{Il r'Hamiltoniana}
ciao

%%%%%%%%%%%%%%%%%%%%%%%%%%%%%%%%%%%%%%%%%%%%%%%%%%%%%%%%%%%%%%%%%%%%%%%%%%%%%%%%%%%%%%%%%
% SECTION RECENT WORKS 
%%%%%%%%%%%%%%%%%%%%%%%%%%%%%%%%%%%%%%%%%%%%%%%%%%%%%%%%%%%%%%%%%%%%%%%%%%%%%%%%%%%%%%%%%

\section{Recent Works}
\subsection{Il r e l'Hamiltoniana}
ciao

%%%%%%%%%%%%%%%%%%%%%%%%%%%%%%%%%%%%%%%%%%%%%%%%%%%%%%%%%%%%%%%%%%%%%%%%%%%%%%%%%%%%%%%%%
% SECTION CODE DESCRIPTION
%%%%%%%%%%%%%%%%%%%%%%%%%%%%%%%%%%%%%%%%%%%%%%%%%%%%%%%%%%%%%%%%%%%%%%%%%%%%%%%%%%%%%%%%%

\section{Code Description}
\subsection{ciao}
ciao

%%%%%%%%%%%%%%%%%%%%%%%%%%%%%%%%%%%%%%%%%%%%%%%%%%%%%%%%%%%%%%%%%%%%%%%%%%%%%%%%%%%%%%%%%
% SECTION CODE DESCRIPTION
%%%%%%%%%%%%%%%%%%%%%%%%%%%%%%%%%%%%%%%%%%%%%%%%%%%%%%%%%%%%%%%%%%%%%%%%%%%%%%%%%%%%%%%%%

\section{Results}
\subsection{ciao}
ciao
\end{document}
