%% Latex markup and citations may be used here
The only way to simulate the entire lifespan of a star, even with the current computational resources, is through 1D simulations, i.\ e.\ by assuming spherical symmetry. Although this approach has been for some aspects extremely successful, fundamental processes as convection and differential rotation are not resolved. Convection plays a \new{leading} role in certain evolutionary stages of stars, being one of the \new{processes responsible} for chemical mixing and energy transport in the stellar interiors. Currently, the only way to map the macroscopic effects of convection in 1D stellar evolution codes is through parametrized models that are stratification-dependent. A milestone in stellar evolution would be to characterize a general convective boundary by means of Hamiltonian and thermodynamical variables only and predict its behavior over time. In this work a differential study of the convective boundary mixing problem (CBM) was carried out, by means of multidimensional hydrodynamic simulations. Different fluid stratifications and convective velocities were tested, and the consequent boundary mixing were measured.

