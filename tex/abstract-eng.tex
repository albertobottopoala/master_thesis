%% Latex markup and citations may be used here
(abstract in english, at most 200 words. Example: \cite{loremIpsum})
\new{
The only way to simulate the entire lifespan of a star, even with the current computational resources, is through 1D simulations, i.e.\ by simulating sperically symmetric objects. Although this approach has been for some aspects extremely successful, some fundamental processes as convection and differential rotation are not resolved. Convection plays a primary role in certain evolutionary stages of stars, being one of the responsibles for chemical mixing and energy transport in the stellar interiors. The only way to currently map the macroscopic effects of convection in 1D stellar evolution codes in through parametrized models that are stratification-dependent. A milestone in stellar evolution would be to characterize a general convective boundary by means of hamiltonian and thermodynamical variables only and predict its behavior over time. In this work a differential study of the convective boundary mixing problem (CBM) was carried out, by means of multidimensional hydrodynamic simulations. Different fluid stratifications and convective velocities were tested, and the consequent boundary mixing measured.
}
