%% Latex markup und Zitate funktionieren auch hier
Sogar mit den heute verfügbaren Rechenkapazitäten sind 1D Simulationen, das heißt sphärisch symmetrische Modelle, die einzige Möglichkeit die gesamte Lebensdauer eines Sterns zu simulieren. Obwohl diese Herangehensweise für machne Aspekte extrem erfolgreich war, können einige fundamentale Prozesse wie Konvektion und differentielle Rotation nicht aufgelöst werden. In gewissen Phasen der Sternentwicklung spielt Konvektion eine essentielle Rolle, da sie für die Durchmischung der chemischen Elemente und den Energietransport im Inneren eines Sterns verantwortlich ist. Momentan besteht der einzige Weg, die makroskopischen Effekte von Konvektion in 1D Sternentwicklungsprogrammen zu erfassen, in der Verwendung von parametrisierten Modellen, welche vom Aufbau der Schichten des Sterns abhängen. Es würde einen Meilenstein der Sternentwicklung darstellen, wenn man eine konvektive Grenzschicht nur mit Hilfe von Hamilton'schen und hydrodynamischen Variablen charakterisieren und ihr zeitliches Verhalten vorhersehen könnte. In dieser Arbeit wurde eine differentielle Studie des Problems der Durchmischung an konvektiven Grenzschichten anhand von mehrdimensionalen hydrodynamischen Simulationen durchgeführt. Es wurden verschiedene Schichtungen des Fluids sowie konvektive Geschwindigkeiten getestet und die daraus folgende Durchmischung an der Grenzschicht ausgewertet.
