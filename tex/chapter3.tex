%%%%%%%%%%%%%%%%%%%%%%%%%%%%%%%%%%%%%%%%%%%%%%%%%%%%%%%%%%%%%%%%%%%%%%%%%%%%%%%%%%%%%%%%%
% SECTION RECENT WORKS 
%%%%%%%%%%%%%%%%%%%%%%%%%%%%%%%%%%%%%%%%%%%%%%%%%%%%%%%%%%%%%%%%%%%%%%%%%%%%%%%%%%%%%%%%%

\chapter{Previous works}

\section{Numerical simulations}

The only way to study thoroughly the macroscopic effects of convection like the boundary mixing problem is through multidimensional simulations. Only in the last decade computational power allowed astrophysics groups to run 3D simulations with a satisfying resolution. We report here what we believe are the most significant works carried out in the past. 

\paragraph{Turbulent Convection in Stellar Interiors I. Hydrodynamic Simulation. \textit{Meakin et al}, 2007.} 
The first extensive work was carried out by C. Meaking and D. Arnett (2007) who simulated the oxygen shell burning and the hydrogen core burning in a $23 M_{\odot}$ star both in 2D and in 3D, using the "box in a star" approach. In the core burning simulation they had only one boundary to analize, during the shell they had both the superior and inferior. This group used the 1D stellar evolution code TYCHO that was then mapped into PROMPI, a multidimensional parallelized hydro code that solves euler equations implementing PPM (piecewise parabolic method) with a nuclear reaction network. They reported $\log A=0.027 \pm 0.38$ and $n=1.05 \pm 0.21$ for the three dimensional case. \\
As we have seen in the previous section, the definition of the bulk-Richardson number is model dependent. In this case in order to define the boundary and its topology, they injected a passive scalar in the convective region. The biggest drop in its concentration defined the boundary topology. They then computed the mean value of the boundary coordinate and its standard deviation, which delimited the initial and final position of the convective boundary over which integrate the Brunt-Väisääla frequency in order to obtain the buoyancy jump. They furthermore defined the length scale of the turbulence as the length where the autocorrelation of the mach number drops below $0.5$. This is a simple but effective set of definitions and procedures that I used in my data analysis too. \\
\paragraph{3D Hydrodynamic Simulations of Carbon Burning in Massive Stars. \textit{Cristini et al}, 2016.} 
Another recent study has been carried out by A. Cristini et al, 2016. They simulated a Carbon burning shell in a $15 M_{\odot}$ star (box in a star method) spanning around $3000$ s and $2 \times 10^9$ cm in four runs with different resolutions (from $128^3$ to $1024^3$). They used the same code PROMPI with self gravity in the Crowling approximation necessary to describe deep interiors of stars, and they used a very similar procedure to get the entrainment rate coefficient. After analyzing the convective structure with help of the Reynolds-averaged Navier-Stokes equations (RANS), they reported parametric values of $A= 0.06 (+0.27 / -0.04)$ and $n= 0.81 (+0.38 / -0.28)$. Although the second one agrees pretty well with the previous study, the first one is not even in the same order of magnitude and affected by huge uncertainty, hence the motivation for this work.
