%%%%%%%%%%%%%%%%%%%%%%%%%%%%%%%%%%%%%%%%%%%%%%%%%%%%%%%%%%%%%%%%%%%%%%%%%%%%%%%%%%%%%%%%%
% SECTION INTRODUCTION
%%%%%%%%%%%%%%%%%%%%%%%%%%%%%%%%%%%%%%%%%%%%%%%%%%%%%%%%%%%%%%%%%%%%%%%%%%%%%%%%%%%%%%%%%


\chapter{Introduction}
Nature in known for having three mechanisms to transport heat.  \\
The first one is through \textbf{heat conduction}. Suppose we have two identical bricks of iron at two different temperatures and we could look at the atomic scale, the only difference we would notice is that the atoms of the warmer brick shake more. To say it in a more rigorous way, their mean kinetic energy (with a randomized velocity) is higher, hence so is the thermal energy. Imagine we put the two bricks in contact, the atoms of the warmer would start to hit and bounce off the atoms of the colder, transferring thermal energy (or heat) from the one to the other, until they reach a situation of equilibrium. This is of course the \textbf{Principle Zero of Thermodynamics}. \\
The second mechanism is \textbf{radiation}. Every object with a non-zero absolute temperature emits a black body radiation, as long as atomic bounds are complex enough in order to allow the application of the Boltzmann statistics for the energetic distribution of electrons (this is true for objects that have a number of atoms on the order of the Avogadro Number, and even far less). This allows a body to lose thermal energy over time through emission of electromagnetic radiation. This is the mechanism through which Earth and other planets re-emit the radiation received by the Sun and reach thermodynamic equilibrium (it cannot happen through conduction, since this would require that molecules in the atmosphere transfer their thermal energy to molecules or atoms in space, which are too rarefied to allow this). \\
The third mechanism, which is the topic of this thesis, is \textbf{convection}. If we consider a fluid stratification like in the atmosphere in which certain thermodynamic conditions are fulfilled, the medium might become unstable in certain regions and give rise to macroscopic ascending and descending blobs which later decay to smaller scale blobs and finally into turbulence. This enables a very efficient transfer of quantities such as heat and chemical composition through different layers of the fluid stratification. \\
A \textit{Cumulonimbus}, the typical cloud of thunderstorms, is a perfect example of such a phenomenon. They seldom reach an altitude higher that ten thousand meters because thermodynamical conditions for convection are generally prohibitive in these regions of the atmosphere. Nevertheless when a convective blob hits the stable layer, two phenomena happen. First of all its turbulent motion erodes mass from it and over time the convective boundary moves upward. Second of all the hit of the blob imprints an internal mode in the stable layer. These two are the reasons for which planes avoid flying not only into thunderstorms but also above them, because they expect to find turbulence. \\
These phenomena happen also in stellar inertia. A convective region over time entrains mass from the stable one, mixing energy and chemical composition (and hence affecting nuclear reaction rates) and imprints internal modes that can be observed on stellar surfaces (the field that studies this phenomenon is asteroseismology, which relies on data of space observatory such as \textit{KEPLER}). It is therefore of fundamental importance to understand the dynamic of the convective boundary and the \textbf{Convective Boundary Mixing} (CBM) problem, in order to properly understand stellar evolution. This is the ultimate goal of this work.

